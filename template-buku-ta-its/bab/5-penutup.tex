\chapter{PENUTUP}
\label{chap:penutup}

% Ubah bagian-bagian berikut dengan isi dari penutup

\section{Kesimpulan}
\label{sec:kesimpulan}

Berdasarkan hasil penelitian dari penggunaan model conditional GAN untuk merekonstruksi sinyal B-scan GPR pada objek berbentuk irregular, didapatkan kesimpulan sebagai berikut:

\begin{enumerate}[nolistsep]

  \item Hasil prediksi model sudah mampu menunjukkan posisi dan ukuran objek dari \emph{ground truth}.

  \item Bentuk dari objek dengan bentuk yang lebih kompleks dan lebih besar masih belum sesuai dengan \emph{ground truth} yang diharapkan. 

  \item Model conditional GAN sangat berpotensi dalam proses transisi dari gambar B-scan GPR menjadi gambar prediksi objek bawah permukaan.

\end{enumerate}

\section{Saran}
\label{chap:saran}

Untuk pengembangan lebih lanjut pada penelitian tugas akhir ini, terdapat beberapa beberapa saran yang dapat dilakukan, antara lain:

\begin{enumerate}[nolistsep]

  \item Memperbanyak data train untuk variasi bentuk dan ukuran benda guna memaksimalkan hasil prediksi gambar.

  \item Menambah jenis objek dan media perambatan baru untuk sinyal GPR hasil simulasi gprMax.

  \item Mengembangkan penelitian untuk bentuk sinyal GPR lainnya, yaitu A-scan maupun C-scan.

\end{enumerate}
