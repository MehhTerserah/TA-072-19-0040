\chapter{PENUTUP}
\label{chap:penutup}

% Ubah bagian-bagian berikut dengan isi dari penutup

\section{Kesimpulan}
\label{sec:kesimpulan}

Berdasarkan hasil penelitian dari penggunaan model conditional GAN untuk merekonstruksi sinyal B-scan GPR pada objek berbentuk irregular, didapatkan kesimpulan sebagai berikut:

\begin{enumerate}[nolistsep]

  \item Selama proses pelatihan model GAN, grafik Total Generator Loss mengalami penurunan nilai dari awal hingga akhir, sedangkan grafik Total Discriminator Loss mengalami penurunan nilai di awal, kemudian mengalami penaikan nilai di ahir. 
  Hal ini menunjukkan bahwa Generator semakin baik dalam menghasilkan gambar yang diinginkan. Namun, Discriminator awalnya semakin baik dalam membedakan data asli dengan data palsu, namun pada akhir pelatihan semakin kesulitan dalam membedakannya.

  \item Berdasarkan hasil evaluasi matriks dari 40 test data GPR ($RMS_{rerata}$ = 0.247, $MSE_{rerata}$ = 0.070, $SSIM_{rerata}$=0.833), model GAN bekerja lebih baik dalam mensintesis gambar struktur bawah permukaan untuk objek dengan posisi dekat dengan permukaan (RMS = 0.239, MSE = 0.63, SSIM = 0.841), berbentuk lingkaran (RMS = 0.237, MSE = 0.064, SSIM = 0.843) dan berbentuk segi empat (RMS = 0.219, MSE = 0.058, SSIM = 0.838).
  Namun, berdasarkan hasil evaluasi visual, output model cukup mampu menunjukkan posisi objek, dan masih belum bisa menunjukkan bentuk dan ukuran objek untuk setiap variasinya.

\end{enumerate}

\section{Saran}
\label{chap:saran}

Untuk pengembangan lebih lanjut pada penelitian tugas akhir ini, terdapat beberapa beberapa saran yang dapat dilakukan, antara lain:

\begin{enumerate}[nolistsep]

  \item Memperbanyak data train untuk variasi bentuk dan ukuran benda guna memaksimalkan hasil prediksi gambar.

  \item Menambah jenis objek dan media perambatan baru untuk sinyal GPR hasil simulasi gprMax.

  \item Mengembangkan penelitian untuk bentuk sinyal GPR lainnya, yaitu A-scan maupun C-scan.

\end{enumerate}
