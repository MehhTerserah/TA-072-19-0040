\begin{center}
  \large\textbf{ABSTRACT}
\end{center}

\addcontentsline{toc}{chapter}{ABSTRACT}

\vspace{2ex}

\begingroup
% Menghilangkan padding
\setlength{\tabcolsep}{0pt}

\noindent
\begin{tabularx}{\textwidth}{l >{\centering}m{3em} X}
  \emph{Name}     & : & \name{}         \\

  \emph{Title}    & : & \engtatitle{}   \\

  \emph{Advisors} & : & 1. \advisor{}   \\
                  &   & 2. \coadvisor{} \\
\end{tabularx}
\endgroup

% Ubah paragraf berikut dengan abstrak dari tugas akhir dalam Bahasa Inggris
\emph{Ground Penetrating Radar (GPR) has often been used in geophysics to detect objects buried underground. 
Not only in soil media, other media such as wood, concrete and road can also be used. 
The detected object can also be in various shapes and materials, one of which is the cavities. 
Cavities are some air-space that arise in concrete due to air trapped in the casting process. 
By using the GPR signal, efforts to detect these cavities can be carried out. 
In this research, the signal reconstruction process from the GPR B-Scan was carried out. 
The signal form that will be used is the ricker wavelet, because the shape is very good at forming seismic data. 
The signal completion process generally uses the Fourier transform when integrating A-Scan signals, so it is relatively complicated and long. 
By applying the conditional generatif adversial network (CGAN), GPR B-Scan data can be synthesized using the Generator and Discriminator functions. 
These data can be used in identifying and classifying objects in the GPR signal, which is the focus of this research in the form of irregular shape cavities in concrete. 
With this research, it is expected to form a method that can reconstruct the GPR B-Scan signal to make it simpler, which can then detect irregular shape cavities in concrete.}

% Ubah kata-kata berikut dengan kata kunci dari tugas akhir dalam Bahasa Inggris
\emph{Keywords}: \emph{Irregular}, \emph{GPR}, \emph{B-Scan}, \emph{CGAN}.
