\begin{center}
  \large\textbf{ABSTRAK}
\end{center}

\addcontentsline{toc}{chapter}{ABSTRAK}

\vspace{2ex}

\begingroup
% Menghilangkan padding
\setlength{\tabcolsep}{0pt}

\noindent
\begin{tabularx}{\textwidth}{l >{\centering}m{2em} X}
  Nama Mahasiswa    & : & \name{}         \\

  Judul Tugas Akhir & : & \tatitle{}      \\

  Pembimbing        & : & 1. \advisor{}   \\
                    &   & 2. \coadvisor{} \\
\end{tabularx}
\endgroup

% Ubah paragraf berikut dengan abstrak dari tugas akhir
\emph{Ground Penetrating Radar} (GPR) sudah sering digunakan dalam ilmu geofisika untuk mendeteksi objek tertimbun bawah tanah. 
Tidak hanya pada media tanah, media lain seperti kayu, beton, dan aspal juga dapat digunakan. 
Objek yang dideteksi juga bisa dalam berbagai bentuk dan materi, salah satunya yaitu \emph{cavities}. 
\emph{Cavities} merupakan rongga udara yang timbul pada beton akibat udara yang terjebak pada proses pengecoran. 
Umumnya bentuk \emph{cavities} adalah irregular. 
Dengan menggunakan sinyal GPR, upaya mendeteksi \emph{cavities} tersebut dapat dilakukan. 
Dalam penelitian ini dilakukan proses rekonstruksi sinyal dari \emph{B-Scan} GPR. 
Bentuk sinyal yang akan digunakan yaitu \emph{ricker wavelet}, karena bentuknya sangat bagus dalam pembentukan data seismik. 
Proses rekonstruksi sinyal umumnya menggunakan transformasi fourier pada saat mengintegrasikan sinyal \emph{A-Scan}, sehingga relatif rumit dan lama. 
Dengan menerapkan \emph{Conditional Generative Adversarial Network} (CGAN), data \emph{B-Scan} GPR dapat disintesis menggunakan fungsi Generator dan Diskriminator. 
Data tersebut dapat digunakan dalam mengidentifikasi dan mengklasifikasi objek pada sinyal GPR, yang fokusan penelitian ini berupa bentuk irregular \emph{cavities} pada beton. 
Dengan penelitian ini, diharapkan terbentuknya suatu metode yang dapat dapat merekonstruksi sinyal \emph{B-Scan} GPR agar lebih sederhana, yang kemudian dapat mendeteksi bentuk irregular \emph{cavities} pada beton.

% Ubah kata-kata berikut dengan kata kunci dari tugas akhir
Kata Kunci: Irregular, GPR, \emph{B-Scan}, CGAN.
