% Atur variabel berikut sesuai namanya

% nama
\newcommand{\name}{John Parulian Siahaan}
\newcommand{\authorname}{Siahaan, John Parulian}
\newcommand{\nickname}{John}
\newcommand{\advisor}{Dion Hayu Fandiantoro, S.T., M.Eng.}
\newcommand{\coadvisor}{Arief Kurniawan, S.T, M.T.}
\newcommand{\examinerone}{Dr. Galileo Galilei, S.T., M.Sc}
\newcommand{\examinertwo}{Friedrich Nietzsche, S.T., M.Sc}
\newcommand{\examinerthree}{Alan Turing, ST., MT}
\newcommand{\headofdepartment}{Dr. Supeno Mardi Susiki Nugroho, S.T., M.T.}

% identitas
\newcommand{\nrp}{0721 19 4000 0040}
\newcommand{\advisornip}{1994202011064}
\newcommand{\coadvisornip}{19740907200212 1 001}
\newcommand{\examineronenip}{18560710 194301 1 001}
\newcommand{\examinertwonip}{18560710 194301 1 001}
\newcommand{\examinerthreenip}{18560710 194301 1 001}
\newcommand{\headofdepartmentnip}{19700313199512 1 001}

% judul
\newcommand{\tatitle}{REKONSTRUKSI SINYAL \emph{GROUND-PENETRATING RADAR} DENGAN MENGGUNAKAN \emph{GENERATIVE ADVERSARIAL NETWORK} UNTUK OBJEK BERBENTUK IRREGULAR YANG TERPENDAM PADA BETON}
\newcommand{\engtatitle}{\emph{RECONSTRUCTION OF GROUND PENETRATING RADAR SIGNALS USING GENERATIVE ADVERSARIAL NETWORK FOR IRREGULAR-SHAPED OBJECTS BURIED IN CONCRETE}}

% tempat
\newcommand{\place}{Surabaya}

% jurusan
\newcommand{\studyprogram}{Teknik Komputer}
\newcommand{\engstudyprogram}{Computer Engineering}

% fakultas
\newcommand{\faculty}{Teknologi Elektro dan Informatika Cerdas}
\newcommand{\engfaculty}{Intelligent Electrical and Informatics Technology}

% singkatan fakultas
\newcommand{\facultyshort}{FTEIC}
\newcommand{\engfacultyshort}{ELECTICS}

% departemen
\newcommand{\department}{Teknik Komputer}
\newcommand{\engdepartment}{Computer Engineering}

% kode mata kuliah
\newcommand{\coursecode}{EC224801}
